This document has been prepared to help you approaching Latex as a formatting tool for your Travlendar+ deliverables. This document suggests you a possible style and format for your deliverables and contains information about basic formatting commands in Latex. A good guide to Latex is available here \href{https://tobi.oetiker.ch/lshort/lshort.pdf}{https://tobi.oetiker.ch/lshort/lshort.pdf}, but you can find many other good references on the web. 

Writing in Latex means writing textual files having a \texttt{.tex} extension and exploiting the Latex markup commands for formatting purposes. Your files then need to be compiled using the Latex compiler. Similarly to programming languages, you can find many editors that help you writing and compiling your latex code. Here \href{https://beebom.com/best-latex-editors/}{https://beebom.com/best-latex-editors/} you have a short oviewview of some of them. Feel free to choose the one you like.  

Include a subsection for each of the following items\footnote{By the way, what follows is the structure of an itemized list in Latex.}:
\begin{itemize}
\item
Purpose: here we include the goals of the project
\item
Scope: here we include an analysis of the world and of the shared phenomena
\item
Definitions, Acronyms, Abbreviations
\item
Revision history
\item
Reference Documents 
\item
Document Structure
\end{itemize}
Below you see how to define the header for a subsection.
\subsection{Purpose}
The main goal of Safestreets is to provide authorities with a tool to control the traffic violations and, in particular, parking violations. The role of citizens is crucial because they send pictures of violations using the system. Safestreets have to store these information and elaborate it before notify authorities.
The elaboration of the information is focused on retrieving some specific data such as:
\begin{itemize}
	\item
	the license plate of the cars involved in the violations
	\item
	the addresses of the events
	\item 
	the streets and the cars with the highest number of violations
	\item 
	the most unsafe areas and the possible solutions to improve the situation
\end{itemize}
In addition Safestreets offers to the municipality the data in order to generate traffic tickets. Furthermore the traffic tickets informations are used to build statistics regarding the effective impact of this initiative.
An other important aspect is that the application need to ensure the chain of custody of the information coming from the user.
\subsubsection{Goals}
\begin{itemize}
	\item
	[G1] allow a user to send photos and information about a traffic violations
	\item 
	[G2] elaborate received data and generate notifications
		\subitem 
		[G2.1] retrive the license plates from photos
		\subitem
		[G2.2] retrieve addresses
	\item
	[G3] calculate statistics
		\subitem
		[G3.1] locate the most unsafe areas
		\subitem
		[G3.2] show the impact of the application in terms of number of violations
		\subitem
		[G3.3] generate other type of statistics like the most egregious offenders
\end{itemize}
\subsection{Scope}
The scope where Safestreets works is tipically the urban enviroment. The main events caused by the world are violations, authorities requiring for statistics and accidents.
The shared phenomena consist in the photos arriving form users, the notifications, the presentation of data and statistics and the municipality sending accindents information.  
The main actors that interact with the application are users, municipality and authorities in general.
%what you write here is a comment that is not shown in the final text