\subsection{Purpose}
The purpose of the DD is to provide a strategic guide to develop the project in the correct way, avoiding the errors and the waste of time that could be generated without studiyng the problem and the possible solutions. The document will be given to the development team.
\subsubsection{Scope}
The web platform will provide a system that manage the reports of the parking violations and elaborates the data to retrieve information about streets and cars:
\begin{itemize}
	\item
	Report violation
	\item 
	Report validation
	\item
	Update statistics
	\item
	Manage streets data
	\item
	Different usage between end user and authority
\end{itemize}
\subsection{Definitions, Acronyms, Abbreviations}
\subsubsection{Definitions}
\begin{itemize}
	\item 
	End-user: The end-user is the common user of the application, that take the picture of the violations and send it with all the report data to the Application server.
	\item 
	Authority: The authority are the municipality that will provide the traffic tickets to who commit a violation and can access different data than the end-user.
	\item 
	Web Server: A Web server uses HTTP (Hypertext Transfer Protocol) to provide the files that compose the Web pages to users, in response to their requests, which are forwarded by their hosts' HTTP clients. Dedicated computers and appliances may be referred to as Web servers as well.
	\item 
	Application Server: An application server is a type of server designed to install, operate and host applications and associated services for end users, IT services and organizations.
\end{itemize}
\subsubsection{Acronyms}
\begin{itemize}
	\item
	API: Application Programming Interface
	\item
	DD: Design Document
	\item
	JDBC: Java DataBase Connectivity
	\item
	RASD: Requirements Analysis and Specifications Document
	\item
	RMI: Remote Method Invocation
	\item
	UX: User Experience
	\item
	ANPR: Automatic number-plate recognition, is the portion of the system
	that is able to recognize the license plate number from the pictures prodived
	by users.
	\item
	CUU: Code Unique Unitary
	
\end{itemize}

\subsubsection{Abbreviations}